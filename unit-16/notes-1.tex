\subsection{The Internet}

The Internet consists of billions of connected computing devices.
End systems are also known as `hosts' and run network applications.
Computers are connected via communication links, which can be fibre optic, copper, radio or satellite.
End systems and different networks are connected to routers and packet switches that forward chunks of data known as `packets'.

The Internet is a network of networks and interconnected Internet service providers (ISPs).
The sending and receiving of messages is controlled by protocols such as TCP, IP, HTTP, Skype and 802.11.
A set of standards exists to help manage the Internet.
Proposals for standards and protocols are published as requests for comments (RFCs), which are made available to key stakeholders to be reviewed and standardised.
The Internet Engineering Taskforce (IETF) is the governing body that oversees the development of the Internet.

The Internet is also an infrastructure that provides services to applications.
These services and applications include web, VoIP, email, games, e-commerce and social networks.
It also provides a programming interface to applications.
Web hooks allow applications to connect to the Internet.

Residential, institutional and mobile access networks each have end systems connected to an edge router.
The bandwidth of a connection is affected by the hardware and connection used.
Radio links are shared, whereas an Ethernet cable is dedicated.

Client and server hosts, and servers in data centres form the network edge.
Access networks are the communication links between devices.
These may be wired or wireless.
The interconnected routers that connect the access networks are known as the `network core'.

Hosts break application messages into smaller chunks known as `packets' and transmit them to the access network.
The speed at which the packet is transmitted to the access network is the link transmission rate, which is also known as `link capacity' or `link bandwidth'.
The packet transmission delay when a packet of length \( L \)~bits is transmitted into a link with transmission rate \( R \)~bits per second is \( \frac{L}{R} \)~seconds.

The network core forwards packets from one router to the next across links that form a path between source and destination.
Each packet is transmitted at full link capacity.
Routing algorithms in the network core determine the source to destination route taken by packets.
Routers forward packets from their input to their appropriate output.

\subsection{Distributed Systems}

A distributed system is one in which hardware or software components located at networked computers communicate and coordinate their actions by only passing messages.
Distributed networks allow for high concurrency.
There is no global notion of the correct time in a distributed network; there is only relative time between computers.
Each component in a network may fail independently.
The failure of one component does not affect the other components.

\subsection{Internet, Intranet and Firewalls}

The Internet is a distributed system that enables users all over the world to make use of its services.
Some highly connected links in the Internet are known as backbones.
These have high link capacity and can be connected via satellite link.

An intranet is part of the Internet that is separately administered and uses a firewall to enforce its own local security policies.
Users in an intranet share data by means of file services.

A firewall is a network security system that monitors and controls incoming and outgoing network traffic according to predetermined security rules.
A firewall establishes a barrier between a trusted internal network and an untrusted external network, such as the Internet.
Firewalls may be network-based or host-based, including network-layer or packet filters and application-layer firewalls.

\subsection{Network Principles}

Networking is concerned with sending messages over a carrier.
\begin{itemize}
  \item Latency is the term given to any kind of delay that occurs during data communication over a network.
  \item Bandwidth is the transmission capacity of a computer network or telecommunication system.
  \item Speed is the rate at which data is able to move.
\end{itemize}

A logical unit of data transmitted via a network is known as a message.
A message of arbitrary length is divided into packets before transmission.
A packet is a bit stream of restricted length.
It includes not only the data, but also relevant addressing information.

\subsection{Switching Schemes}

A network is a set of nodes connected by circuits.
In order to transmit information between two nodes, a switching scheme is required.
\begin{itemize}
  \item Broadcast --- no switching, data is transmitted throughout the network
  \item Circuit switching --- source and destination are connected through a switch at an exchange
  \item Packet switching --- a store-and-forward network with a computer at each end
\end{itemize}

\subsubsection{Circuit Switching}

End-to-end resources are allocated to create a reserved connection between source and destination.
The resources are dedicated; there is no sharing.
Full bandwidth circuit-like performance is guaranteed.
Circuit segments remain idle if not used for a connection.
This transmission scheme is traditionally used in telephone networks.

\subsubsection{Packet Switching}

Packet switching is a store-and-forward transmission scheme that allows more users to use the network.
Packets are transmitted to a router, where they are stored in a queue, waiting for an output link.
The entire packet must arrive at the router before it can be transmitted through the next link.

The delay for transmission of an \( L \)~bit packet through an \( R \)~bits per second link is \( \frac{L}{R} \)~seconds.
This is the `one-hop transmission delay'.
Assuming no propagation delay, the end-to-end delay for transmission from source to destination through one router is \( 2 \frac{L}{R} \).
In general, the delay is \( N \frac{L}{R} \), where \( N \) is the number of links and, therefore, \( N - 1 \) is the number of routers.

If the arrival rate to a router exceeds its transmission rate for a period of time, the packets will queue, waiting to be transmitted through a link.
Additionally, packets can be dropped (lost) if the memory (buffer) is filled.

\subsection{Internet Structure (Network of Networks)}

End systems connect to the Internet via access ISPs.
These may be residential, company or university ISPs.
In turn, access ISPs must be interconnected so that any two hosts can communicate.
The resulting network is very complex.
Its evolution has been driven by economics and national policies.

Connecting each access ISP to every other access ISP is not feasible.
This would require \( \function{O}{n^2} \) connections.
An alternative is to connect each access ISP to a global ISP\@.
There would be an economic agreement between customer and provider ISPs.
However, if one global ISP is a viable business, there will exist competitors, which must in turn be interconnected.

Such global ISPs are connected to each other through Internet exchange points (IXPs) or peering links.
Regional networks also exist to connect groups of access networks to global ISPs.
Large Internet companies, such as Google and Microsoft, run their own content delivery networks to bypass connections and provide services and content closer to end systems.

The result is small number of well-connected large networks.
These are `Tier~1' commercial ISPs that offer national and international coverage, and private content delivery networks that connect data centres to the Internet to bypass Tier~1 and regional ISPs.
