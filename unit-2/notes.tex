\subsection{Anatomy of a Computer System}

A computer is a complex system (machine) that can be instructed to carry out sequences of arithmetic or logical operations automatically via computer programming.

A computing device must be able to
\begin{itemize}
  \item load a program (input interface),
  \item process instructions in the correct order (track progress, storage and decoding of instructions),
  \item access data according to its instructions (local storage),
  \item perform computations (calculation `engine'),
  \item make decisions according to its computations (control mechanism), and
  \item send results to an external device (output interface).
\end{itemize}

Thus, all computers, regardless of their implementing technology, have five basic subsystems.
\begin{enumerate}
  \item Memory
  \item Control unit
  \item Arithmetic logic unit (ALU)
  \item Input unit
  \item Output unit
\end{enumerate}

\subsubsection{Memory}

Memory locations have finite capacity.
Data may not fit in a memory location.
Blocks of four or eight bytes are used so often as a unit that they are known as memory `words`.

Computer memory is known as random access memory (RAM).
This simply means that the computer can refer to memory locations in any order.

\subsubsection{Control unit}

The control unit of a computer is where the fetch/execute cycle occurs.
It fetches a machine instruction from memory and performs other operations of the fetch/execute cycle accordingly.

\subsubsection{Arithmetic Logic Unit (ALU)}

The arithmetic logic unit carries out each machine instruction with a separate circuit.
It contains various circuits for arithmetic and logic, such as addition, subtraction, multiplication, comparison and logic gates.

\subsubsection{Input and Output Units}

These components are the wires and circuits through which information travels into and out of a computer.
Without input or output, a computer is useless.

Peripherals connect to the computer through input/output (I/O) ports.
They are not considered parts of the computer.

\subsection{The Fetch-Execute Cycle}

A program is loaded into memory and the address of the first instruction is placed in the program counter (PC).

The fetch-execute cycle proceeds as follows.
\begin{enumerate}
  \item Instruction fetch (IF)
  \item Instruction decode (ID)
  \item Data fetch (DF) / operand fetch (OF)
  \item Instruction execution (EX)
  \item Result return (RR) / store (ST)
\end{enumerate}

\subsubsection{Instruction Fetch (IF)}

The instruction at the memory address given by the PC is copied to the instruction register of the control unit.
The PC is incremented to point at the next instruction to be fetched.

\subsubsection{Instruction Decode (ID)}

The ALU is prepared for the operation specified by the instruction.
The decoder finds the addresses of the instruction operands.
These addresses are passed to the ALU circuit that fetches the operands from memory in the next stage.
The decoder also finds the destination address for the result.
This is passed to the RR circuit.

\subsubsection{Data fetch (DF)}

The operands are copied from the specified memory addresses into the ALU circuits.
The data remains in memory and is not destroyed.

\subsubsection{Instruction Execute (EX)}

The ALU performs the operation on its operands.
The result is held in its circuitry.

\subsubsection{Result Return (RR)}

The result of the EX stage is stored at the specified destination memory address.
The cycle begins again.

\subsection{Machine Instructions}

A computer ``knows'' very few instructions.
The decoder hardware in the control unit recognises, and the ALU performs, of the order of \num{100} instructions.
Everything that a computer does must be reduced to some combination of these primitive instructions.
Computers can carry out millions of these instructions per second.
