\subsection{Processes}

A process is a program in execution.
A program is a passive entity stored on the disk, whereas a process is an active entity.
A program becomes a process when its executable file is loaded into memory.

A program may be started through a GUI or CLI\@.
One program may consist of several processes.
For example, there may be one process for each user or instance.

Processes are represented by three main components.
\begin{itemize}
  \item Executable program code (also known as the `text segment')
  \item Data related to the program
  \item Execution context
  \begin{itemize}
    \item process ID, group ID, user ID
    \item stack pointer, program counter, CPU registers
    \item file descriptors, locks, network sockets
  \end{itemize}
\end{itemize}
The execution context is essential for process switching.

A process can be an execution environment for other code.
For example, the JVM executes as a process that interprets loaded Java code and performs instructions on behalf of it.

\subsection{Process Structure}

A process is more than just the program code.
It also includes current activity such as the value of the program counter and the contents of CPU registers.
The process stack contains temporary data, such as function parameters, return addresses and local variables.
The data segment contains global variables and a heap --- memory allocated dynamically at runtime.

The layout of the memory of a process is known as a `process image'.
A process image is divided into segments.
\begin{itemize}
  \item Stack segment
  \begin{itemize}
    \item function parameters, return addresses and local variables
  \end{itemize}
  \item Data segment
  \begin{itemize}
    \item static variables and constants
    \item memory allocated dynamically from the heap
  \end{itemize}
  \item Text segment
  \begin{itemize}
    \item program code --- shared between processes
  \end{itemize}
\end{itemize}

\begin{figure}[htp]
  \centering
  \begin{tabular}{r|c|l}
    \hline
    max & Stack          & \multirow{2}{*}{Stack segment} \\
        & \(\downarrow\) &                                \\
    \cline{2-3}
        & \(\uparrow\)   & \multirow{3}{*}{Data segment}  \\
        & Heap           &                                \\
    \cline{2-2}
        & Data           &                                \\
    \cline{2-3}
        & Program code   & Text segment                   \\
    \cline{2-3}
      0 & Kernel         & Protected                      \\
    \hline
  \end{tabular}
  \caption*{The process image.}
\end{figure}

Each invoked program results in the creation of a separate process with its own image.
Each image appears to cover the entire address space.
In reality, these are virtual addresses mapped to physical addresses.

\subsection{Process Control Block (PCB)}

The information associated with each process is stored as a process control block (PCB), which is also known as a task control block (TCB).
This information includes
\begin{itemize}
  \item process state,
  \item program counter (PC) --- address of next instruction to be executed,
  \item CPU registers,
  \item CPU scheduling information --- priorities and scheduling queue pointers,
  \item memory-management information,
  \item accounting information --- CPU usage, time elapsed and time limits, and
  \item I/O status information --- I/O devices allocated to the processes and list of open files.
\end{itemize}

A process with multiple threads of execution has instructions at multiple addresses executing at once.
This requires multiple program counters.
These additional thread details are stored in the PCB.

\subsection{Process States}

As a process is executed, it changes state.
Possible states for a process are
\begin{itemize}
  \item new --- the process is being created,
  \item ready --- the process is waiting to be assigned to a processor,
  \item running --- the process is executing instructions,
  \item waiting (blocked) --- the process is waiting for an event to occur, and
  \item terminated --- the process has finished execution.
\end{itemize}

When a process is admitted to the system, its state changes from `new' to `ready'.
When it is dispatched to a CPU, its state changes to `running'.
From there, it may time out, in which case its state reverts to `ready', or it may need to wait for an event to occur, in which case its state changes to `blocked'.
Alternatively, it may be released from execution, in which case its state changes to `terminated'.
A process in the `blocked' state moves to the `ready' state once the event it is waiting for occurs.

\subsection{Process Scheduling}

Process scheduling is used to maximise CPU usage and minimise response time by quickly switching processes onto the CPU for time sharing.
A process will `give up' the CPU if it sends an I/O request or after a certain period of time as elapsed.
When a process gives up the CPU, it is added to the ready queue or a device queue.
The process scheduler selects available processes in the ready queue for the next execution.

The OS maintains various scheduling queues.
\begin{itemize}
  \item Job queue --- all processes in the entire system
  \item Ready queue --- all processes in main memory ready and waiting to execute
  \item Device queues --- all processes waiting for an I/O device
\end{itemize}

Processes migrate between queues.
Queuing diagrams are used to represent flows of processes and resources.

\subsubsection{Types of Scheduler}

A short-term scheduler (or CPU scheduler) selects which process should be executed next and allocates a CPU\@.
Sometimes this is the only scheduler in a system.
It is invoked frequently and must be fast.

A long-term scheduler (or job scheduler) selects which processes should be brought into the ready queue.
This is invoked infrequently and may be slow.
The long-term scheduler controls the degree of multiprogramming.

\subsubsection{Types of Process}

An I/O-bound process spends more time performing I/O than computations.
It uses many short CPU bursts.

A CPU-bound process spends more time performing computations.
It uses a few long CPU bursts.

The long-term scheduler attempts to schedule a good process mix.

\subsubsection{Context Switching}

When the CPU switches to another process, the system saves the state of the old process and loads the saved state of the new process via a context switch.
The context of a process is represented by its PCB.

The time required to perform a context switch is pure overhead; the system does no useful work while switching.
More complex operating systems and process control blocks require more time for a context switch.
The time required is dependent upon hardware support.
Some hardware provides multiple sets of registers per CPU\@.
This allows multiple contexts to be loaded simultaneously.

\subsubsection{Dispatch Events}

Context switches (dispatch events) are triggered by clock interrupts.
These occur after a set time interval, typically between \SIlist{3;10}{\milli\second}.
The execution of processes is interrupted and control returns to the OS\@.
The current process is moved to the ready queue.
The frequency of such an interrupt is an important system parameter used to balance the overhead of context switching and the responsiveness of the system.

Context switches are also triggered by I/O interrupts.
When an I/O event occurs and data has been loaded into memory, the current process is interrupted and all blocked processes waiting for the I/O event are moved to the ready queue.
The dispatcher must decide whether to continue execution of the interrupted process.

Memory faults or page faults also trigger context switches.
These occur when an executing process refers to a virtual memory address that has not been allocated a physical memory location (the data requested is still on the disk).
The current process is interrupted and an I/O request for the data is issued.
The current process is moved to the blocked state and execution switches to another process from the ready queue.
When the I/O has completed, the blocked process returns to the ready queue.

\subsubsection{Dispatcher}

A dispatcher is an OS function that allocates processes to the CPU and switches the CPU between processes.
The weakness of a dispatcher with a single device queue is that all processes waiting for an event must be transferred from the device queue to the ready queue when an event occurs.
A dispatcher with multiple queues for different device events moves only the processes waiting for that particular event to the ready queue.

\subsubsection{Virtualisation and Swapping}

A computer system may appear to allow an unlimited number of processes to occur concurrently.
Not all of these processes can fit into physical memory.
It is useful, therefore, to achieve medium-term scheduling and reduce the degree of multiprogramming by removing a process from memory.

Moving a process between memory and storage is known as `swapping'.
This requires two additional process states: `ready-suspended' and `blocked-suspended'.

A process in the `ready-suspended' or `blocked-suspended' state is activated to the `ready' or `blocked' state when it is swapped into memory.
It is suspended when it is swapped out of memory.
A process in the `blocked-suspended' state is moved to the `ready-suspended' state when the event for which it is waiting has occurred.
A process in the `running' state may be suspended.

\subsection{Process Creation}

The system must provide mechanisms for process creation and process termination.

Processes are created at system boot, when an existing process spawns a child process, when the user requests the creation of a process, or when a batch systems executes its next job.
When a parent process creates a child process, a tree of processes is formed.
A process is identified and managed via its process identifier (PID).

When a process is created, it is assigned a unique PID, memory is allocated for its process image, its PCB is initialised, and it is added to the ready queue.

\subsubsection{Parent and Child Processes}

A child process may share all the resources of its parent, a subset of those resources or none of them.
A parent and child process may execute concurrently, or the parent may wait for its children to terminate.

A child process is a duplicate of the address space of its parent.
The child process may load a new program into its address space.

\subsubsection{Process Creation in Unix}

A process creates new processes with the kernel system calls \texttt{fork()} and \texttt{exec()}.
\begin{itemize}
  \item A new slot is allocated in the process table.
  \item A unique PID is assigned to the child process.
  \item The process image of the parent is copied, apart from the shared memory areas.
  \item The child process owns the same open files as the parent.
  \item The child process is added to the ready queue.
\end{itemize}

When \texttt{fork()} is called by the parent process, execution switches to kernel mode and the OS creates a copy of the parent process.
This includes the program code in the text segment of the process image.
Thus, the \texttt{fork()} call is also copied.
The parent process continues execution at the return from the \texttt{fork()} call.
The child process begins execution at the return from the \texttt{fork()} call.

In the parent process, the \texttt{fork()} call returns the PID of the child process.
In the child process, the  \texttt{fork()} call returns zero.
This is used by the remainder of the program to identify whether it is running as a parent or a child.
The program may perform different instructions in either of these cases.

For example, the child process may execute the \texttt{exec()} system call, which takes a program name as a parameter.
This system call replaces the content of the child process image with a process image of the specified program.

The \texttt{exec()} system call has a number of variations with different suffixes that describe what parameters are passed to the new process image.
These variations are \texttt{execl()}, \texttt{execle()}, \texttt{execlp()}, \texttt{execv()}, \texttt{execve()} and \texttt{execvp()}.

\begin{itemize}
  \item \texttt{e} --- An array of pointers of environment variables is passed.
  \item \texttt{l} --- Command line arguments are passed individually.
  \item \texttt{p} --- The \texttt{PATH} environment variable is used to find the program to be executed.
  \item \texttt{v} --- Command line arguments are passed as an array of pointers.
\end{itemize}

\subsection{Process Waiting}

A parent may use the system call \texttt{waitpid()} to wait for the termination of a child process.

\subsection{Process Termination}

There are a number of ways in which a process may be terminated.
\begin{itemize}
  \item Normal termination (voluntary) or error termination (voluntary)
  \begin{itemize}
    \item the process reaches regular completion, with or without an error code
  \end{itemize}
  \item Abnormal termination (involuntary)
  \begin{itemize}
    \item OS or user intervenes with kill signal
    \item process attempts to access forbidden memory locations
    \item process times out
    \item process encounters an I/O error
    \item stack overflow
    \item parent process is terminated
  \end{itemize}
\end{itemize}

Some operating systems do not allow a child process to exist if its parent has been terminated.
This is known as `cascading termination'.

A `zombie process' is a process that has terminated but still has an entry in the process table.
This occurs when a child process terminates.
It allows a parent process to read the exit codes of its children.

\subsection{Execution of the Operating System}

An OS may be designed so that it executes
\begin{itemize}
  \item separately from processes (non-process kernel),
  \item as part of the user process images (with mode switching), or
  \item as a set of processes (microkernel, with mode and context switching).
\end{itemize}

\subsubsection{Non-Process Kernel OS}

The kernel acts as a monitor for user processes.
All processes are user processes.
There is no concurrent execution of user processes and the kernel; the kernel may execute only during interruptions.

The OS code is placed in a reserved memory region and executes in privileged mode.
It has its own system stack.

\subsubsection{OS Execution within User Processes}

The user address space includes kernel functions.
System functions may be called from within a process by switching to kernel mode.
No context switching is required as the system functions execute within the same process.
The dispatcher executes context switches outside of processes.

\subsubsection{Process-Based OS}

Kernel functions run in kernel mode as separate processes.
The dispatcher handles context switching separately.
This allows concurrency within the kernel, with kernel processes scheduled together with user processes.
This is useful in multiprocessor environments, as kernel functionality may be distributed amongst CPU cores.
