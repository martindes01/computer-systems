\subsection[Execution on the JVM]{Execution on the Java Virtual Machine}

The \texttt{javac} compiler compiles Java source code in \texttt{.java} files to intermediate-level Java bytecode in \texttt{.class} files.
As long as there is a JVM on another platform, the bytecode can be run on that platform without recompilation.
This feature is known as `portability'.

The JRE loads Java bytecode files, verifies their internal consistency and executes their methods using the JVM.\@
Different CPUs or operating systems can run the same Java bytecode but use different JREs.

\subsection{Stack Frames}

Every time a method is called in Java, a stack frame is constructed for it.
Each frame has
\begin{itemize}
  \item a reference to the frame of its calling method,
  \item space for local variables,
  \item space for an operand stack
  \item a PC and SP (for the JVM, not the CPU), and
  \item other items.
\end{itemize}

\subsubsection{Variable and Operand Storage}

All local variables and operand stack entries are stored as \num{4}~bytes.
The \texttt{long} and \texttt{double} data types, which need \num{8}~bytes, are stored as two consecutive entries.

\begin{table}[htp]
  \centering
  \caption*{Storage of data types.}
  \begin{tabular}{lrrr}
    \toprule
    Data type & \multicolumn{2}{c}{Size} & Slots \\
    \cmidrule(l{0.25em}r{0.25em}){2-3}
    & / \si{\bit} & / \si{\byte} & \\
    \midrule
    \texttt{bool}   & 1 &   & 1 \\
    \texttt{byte}   &   & 1 & 1 \\
    \texttt{char}   &   & 2 & 1 \\
    \texttt{short}  &   & 2 & 1 \\
    \texttt{int}    &   & 4 & 1 \\
    \texttt{float}  &   & 4 & 1 \\
    reference       &   & 4 & 1 \\
    \texttt{long}   &   & 8 & 2 \\
    \texttt{double} &   & 8 & 2 \\
    \bottomrule
  \end{tabular}
\end{table}

\newpage

\subsubsection{Local Variables}

In the context of Java bytecode, local variables include:
\begin{enumerate}
  \item \texttt{this} reference (for non-static methods only)
  \begin{itemize}
    \item a reference to the object on which the method was called
    \item stored as local variable in slot~\num{0}
  \end{itemize}
  \item Parameters of the method
  \begin{itemize}
    \item stored from slot~\num{0} onwards for static methods
    \item stored from slot~\num{1} onwards for non-static methods
  \end{itemize}
  \item Variables declared inside the method
  \begin{itemize}
    \item stored in slots immediately after parameters
  \end{itemize}
\end{enumerate}

Instance variables and class (static) variables are not stored as local variables since they are defined outside of methods.

\subsubsection{Slot Numbers}

Java bytecode does not refer to local variables by their identifiers in the source code.
Instead, it refers to them by their slot numbers.
A local variable of the \texttt{long} or \texttt{double} data types takes the number of the first of its pair of slots.

\begin{figure}[htp]
  \centering
  \begin{tabular}{c || c|c|c|c || c|c|c|c || c|c|c|c || c || c || c}
    \hline
    \ldots &
    \hphantom{1} & \hphantom{2} & \hphantom{3} & \hphantom{4} &
    \hphantom{1} & \hphantom{2} & \hphantom{3} & \hphantom{4} &
    \hphantom{1} & \hphantom{2} & \hphantom{3} & \hphantom{4} &
    \ldots &
    operand stack &
    \ldots \\
    \hline
    \multicolumn{1}{c||}{} &
    \multicolumn{4}{c||}{slot 0} &
    \multicolumn{4}{c||}{slot 1} &
    \multicolumn{4}{c||}{slot 2} &
    \multicolumn{1}{c}{} &
    \multicolumn{1}{c}{} &
    \multicolumn{1}{c}{} \\
  \end{tabular}
  \caption*{Local variable slots in the stack frame.}
\end{figure}

\subsubsection{Stack Overflow}

The operand stack can never overflow.
The Java compiler calculates exactly how much space is required and the loader checks each method to verify that no operand stack underflow or overflow is possible.
Operand stack overflow is different from stack overflow, which occurs when a new frame is needed, but there is not sufficient memory.

\subsection{Java Bytecode Instructions}

Each Java bytecode instruction takes up at least one byte (for its opcode).
It may also have one or more operand bytes.
For each opcode, there is a corresponding mnemonic.

A single operand may consist of multiple bytes.
In this case, the operand is stored with its most significant byte first (in the smaller address).
This order is known as `big endian'.

\subsubsection{Arithmetic}

Each arithmetic instruction has a prefix letter that denotes the data type on which it operates.
For example, the \texttt{iadd} instruction pops two entries from the operand stack, adds them as if they are of the type \texttt{int}, and pushes them back onto the stack.

\begin{table}[htp]
  \centering
  \caption*{Mnemonic prefixes.}
  \begin{tabular}{lll}
    \toprule
    Data type & Prefix & Addition \\
    \midrule
    \texttt{byte}   & \texttt{b} & \\
    \texttt{short}  & \texttt{s} & \\
    \texttt{int}    & \texttt{i} & \texttt{iadd} \\
    \texttt{long}   & \texttt{l} & \texttt{ladd} \\
    \texttt{float}  & \texttt{f} & \texttt{fadd} \\
    \texttt{double} & \texttt{d} & \texttt{dadd} \\
    reference       & \texttt{a} & \\
    \bottomrule
  \end{tabular}
\end{table}

Errors are likely to occur if an instruction is used on the wrong data type.
For example, calling \texttt{fadd} on \texttt{int} entries will result in the wrong answer since the instruction expects entries in floating point representation.
If there is only one operand on the stack and the instruction expects two, operand stack underflow will occur.
This can occur when \texttt{ladd} is called on \texttt{int} entries.

No checks are performed during execution.
The JRE verifies the bytecode when the class is loaded.
This includes checking that types are used consistently.

\subsubsection{Pushing}

The \texttt{bipush <1\_byte\_operand>} instruction has two prefixes: \texttt{b} for \texttt{byte} and \texttt{i} for \texttt{int}.
This instruction sign extends its \texttt{byte} operand (\SI{1}{\byte}) to an \texttt{int} value (\SI{4}{\byte}) and pushes the result to the operand stack.
Similarly, there exists an \texttt{sipush <2\_byte\_operand>} instruction to push a \texttt{short} to the operand stack as an \texttt{int}.

There exist seven instructions for pushing common integer constants to the operand stack that do not require an operand.
Instructions that take no operand help to save space in the bytecode.

\begin{table}[htp]
  \centering
  \caption*{Instructions for pushing common integer constants.}
  \begin{tabular}{lr}
    \toprule
    Instruction & Value pushed \\
    \midrule
    \texttt{iconst\_m1} & -1 \\
    \texttt{iconst\_0}  &  0 \\
    \texttt{iconst\_1}  &  1 \\
    \texttt{iconst\_2}  &  2 \\
    \ldots & \ldots \\
    \texttt{iconst\_5}  &  5 \\
    \bottomrule
  \end{tabular}
\end{table}

Load instructions of the form \texttt{iload <slot\_number>} push local variables (indicated by their slot numbers) onto the operand stack.
There are special load instructions that take no operand for pushing values stored in common slots to the operand stack (\texttt{iload\_0}, \texttt{iload\_1}, \texttt{iload\_2}, \texttt{iload\_3}).

\subsubsection{Popping}

Store instructions pop entries from the operand stack into local variable slots.
For example, \texttt{istore <slot\_number>} or \texttt{istore\_2}.

\subsubsection{Jumps}

In Java bytecode, the source and destination of a jump must be within the same method.
It is not possible to jump to another method.

Unconditional jumps are achieved with the \texttt{goto <2\_byte\_offset>} instruction.
The operand is a \SI{2}{\byte} offset that is added to the address of the instruction to calculate the address of the next instruction to execute.
There exists a version of this instruction that takes a \SI{4}{\byte} operand to allow for larger jumps: \texttt{goto\_w <4\_byte\_offset>}.

Conditional jumps are achieved with instructions of the form \texttt{ifeq <offset>} and \texttt{if\_icmpeq <offset>}.

\subsection{Call and Return}

\subsubsection{The Stack}

If method \texttt{A} calls a static method \texttt{B(p0, p1)} that returns a result:
\begin{enumerate}
  \item method~\texttt{A} calculates the arguments on its operand stack,
  \item a stack frame is constructed for method~\texttt{B},
  \begin{itemize}
    \item the arguments are popped from the operand stack of method~\texttt{A},
    \item the values are used to initialise the local variables \texttt{p0} and \texttt{p1} of method~\texttt{B},
  \end{itemize}
  \item method~\texttt{B} calculates the result on its stack frame
  \item the result is pushed to the operand stack of method~\texttt{A}, and
  \item execution returns to method~\texttt{A}, discarding the stack frame of method~\texttt{B}.
\end{enumerate}

On the operand stack of method~\texttt{A}, this has the effect of

\begin{equation*}
  \ldots, \textnormal{p0}, \textnormal{p1} \quad \rightarrow \quad \ldots, \textnormal{result}
\end{equation*}

Each frame has its own PC.\@
While method~\texttt{B} is under execution, its PC is used.
When execution returns to method~\texttt{A}, it resumes with its old PC value.
The local variables of method~\texttt{A} are unchanged by method~\texttt{B}.

Linked frames have the effect of a return stack.
A chain of linked frames in the JVM is officially known as a stack.

\subsubsection{Runtime Constant Pool}

The class variables, instance variables and methods of a class may be used by other classes.
Since classes are compiled separately, the compiler does not know the address of these items in other classes.
Hence, the class, source file and bytecode file must share the same name to create a `symbolic reference'.

In addition to class variables, instance variables and method definitions, each class in Java has a runtime constant pool that contains read-only constants used in the class.
This includes literal constants, symbolic references to methods and their classes, types and more.

Each pool entry has an index.
Each stack frame includes a pointer to the constant pool of its class in its space for `other items'.

\subsubsection{Static Method Calls}

Static methods are called using the \texttt{invokestatic <2\_byte\_index>} instruction.
The operand is an index in the runtime constant pool of the class of the current method.
The indexed constant pool entry is a symbolic reference to the requested class and method.

The JVM uses this information to find the address of the class, the size needed for the new frame, and the address of the bytecode for the method.

\subsubsection{Method Returns}

The return instruction for void methods is \texttt{return}.
This has no effect on the operand stack of the caller.

For methods that return a result, the instruction is of the form \texttt{ireturn}.
This pops the result from the operand stack of the current frame and pushes it to the operand stack of the caller.

After either of these return instructions, the current frame is discarded and execution continues on the frame of the caller, resuming from its PC.

\subsection{Java Disassembler}

The Java `disassembler' (\texttt{javap}) converts Java bytecode to mnemonics with the \texttt{-c} option.
For example, \texttt{MyClass.class} can be viewed with \texttt{javap -c MyClass}.
Jump instruction operands in disassembler output show the destination instruction address rather than the offset used in the bytecode.
