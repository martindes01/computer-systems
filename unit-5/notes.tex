\subsection{How Subroutines Work}

\subsubsection{Call and Return Instructions}

When execution jumps from line \( n \) to a subroutine, there needs to be a way to instruct the computer to continue execution on line \( n + 1 \) once the subroutine is complete.

Two hypothetical instructions that could be used to achieve this are \texttt{call}, which jumps to the subroutine and stores the return address (the current PC value) somewhere suitable, and \texttt{ret}, which reads the return address from where it was stored and loads it into the PC.

\subsubsection{Method~1 --- Storing Return Address as First Byte}

Using this method, the return address is stored as the first byte of the subroutine.
The instruction \texttt{call N} would store the return address at location \( N \) and begin executing the subroutine at location \( N + 1 \).
The instruction \texttt{ret N} would load the return address stored at location \( N \) into the PC.

The disadvantage of this method is that only one return address can be stored for each subroutine.
Consequently, a subroutine could not call itself, as this would require two return addresses (one for the original call and another for the recursive call).

FORTRAN --- the first commercially available high-level programming language --- disallowed recursion in order to implement this method.

\subsubsection{Introduction to Stacks}

A stack is a data structure that can flexibly store a variable number of bytes.
Stacks employ `last in, first out' (LIFO) semantics.
Elements are `pushed' to or `popped' from the top of the stack.
A stack pointer (SP) is a CPU register that points to the top of the stack.
It is not necessary to know where the bottom of the stack is, but programmers must make sure they do not attempt to pop from an empty stack.

When a value is pushed to the stack, it is written to memory at the address given by the SP.\@
The SP is then incremented.

When a value is popped from the stack, the SP is decremented, then the value is read from memory at the address given by SP.\@
Popped values remain in memory and are later overwritten by pushed values.

\subsubsection{Method~2 --- Storing Return Address on a Stack}

Using this method, the return address is pushed to a stack when the subroutine is entered.
The instruction \texttt{call N} would push the return address to the stack and begin executing the subroutine at location \( N \).
The instruction \texttt{ret} would pop the return address from the stack and load it into the PC.

This method allows multiple return addresses to be stored and, therefore, makes recursion possible.

\subsubsection{Saving Registers}

A subroutine may need to use CPU registers in its calculations.
However, the previous register values will need to be restored when execution returns to the caller.

A common solution to this problem is to push all register values to the stack at the beginning of a subroutine (after the return address has been pushed) and to pop the register values from the stack at the end of the subroutine (before the return address is popped).
Register values are popped in reverse order.

The return address and saved registers are known as a `stack frame'.

\subsection{Stacks for Calculations}

\subsubsection{Reverse Polish Notation (RPN)}

Reverse Polish notation (RPN) or postfix notation is a method of writing calculations in which operators follow their operands and the order of operations is as written (no brackets are needed).
This differs from standard infix notation, in which operators are written between or around their operands and the order of operations is determined by precedence.

RPN is a useful way of implementing calculations with stacks.
If the next item in the input stream is a value or variable, it is pushed to the operand stack.
If the next item is an operation, the relevant number of operands is popped from the operand stack, the operation is performed, and the result is pushed to the operand stack.

The following infix calculation and postfix calculation are equivalent.

\begin{equation*}
  \left( 5 + 2 \right) \times \sqrt{ x \times x + y \times y } + 8
\end{equation*}

\begin{equation*}
  5 \> 2 + x \> x \times y \> y \times + \sqrt{\>} \times 8 \> +
\end{equation*}

One notation that shows the effect of an operation, such as subtraction, on a stack is as follows.

\begin{equation*}
  \ldots, \textnormal{val1}, \textnormal{val2} \quad \rightarrow \quad \ldots, \textnormal{val1} - \textnormal{val2}
\end{equation*}

Java bytecode uses operand stacks for calculations.
In Java, each method call has its own operand stack.

\subsubsection{Stack Machines}

Stack machines use a return stack for subroutine return, and an operand stack for RPN calculations.
Registers are not needed for calculations.
This allows more space for calculations, and machine instructions do not need to specify registers.
However, it is difficult to know where things are on the stack.

The term `operand' could refer either to the byte(s) after the instruction opcode in memory, or the entries in the operand stack.

\subsection{Bitwise Boolean Operations}

The XOR operation is an exclusive OR.\@
It returns true only if exactly one operand is true.

\begin{table}[htp]
  \centering
  \caption*{Truth tables.}
  \begin{tabular}{ccccc}
    \toprule
    A & B & A AND B & A OR B & A XOR B \\
    \midrule
    0 & 0 & 0 & 0 & 0 \\
    0 & 1 & 0 & 1 & 1 \\
    1 & 0 & 0 & 1 & 1 \\
    1 & 1 & 1 & 1 & 0 \\
    \bottomrule
  \end{tabular}
\end{table}

\subsection{Conditional and Unconditional Jumps}

No stacks are used when an unconditional jump is executed.
Conditional jumps are only executed if a condition is true; if the condition is false, execution continues to the next instruction.

Java bytecode contains six comparisons that are used in conditional jump instructions.
There are two types of conditional jump instruction.
Jump instructions of the form \texttt{ifXX} pop one value from the stack and compare it to zero.
Jump instructions of the form \texttt{if\_cmpXX} pop two values from the stack and compare them to each other.

For example, \texttt{ifeq N} pops a value from the stack and jumps to location \( N \) if the value is zero.
The instruction \texttt{if\_cmpeq N} pops two values from the stack and jumps to location \( N \) if they are equal.

\begin{table}[htp]
  \centering
  \caption*{Conditional jump instructions.}
  \begin{tabular}{clll}
    \toprule
    \multicolumn{2}{c}{Comparison} & \multicolumn{2}{c}{Instruction} \\
    Symbol & Mnemonic & \texttt{ifXX} & \texttt{if\_cmpXX} \\
    \midrule
    \( = \)    & \texttt{eq} & \texttt{ifeq} & \texttt{if\_cmpeq} \\
    \( < \)    & \texttt{lt} & \texttt{iflt} & \texttt{if\_cmplt} \\
    \( \leq \) & \texttt{le} & \texttt{ifle} & \texttt{if\_cmple} \\
    \( \neq \) & \texttt{ne} & \texttt{ifne} & \texttt{if\_cmpne} \\
    \( > \)    & \texttt{gt} & \texttt{ifgt} & \texttt{if\_cmpgt} \\
    \( \geq \) & \texttt{ge} & \texttt{ifge} & \texttt{if\_cmpge} \\
    \bottomrule
  \end{tabular}
\end{table}
