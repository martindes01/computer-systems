\subsection{The von~Neumann Architecture}

\subsubsection{Central Processing Unit (CPU)}

The central processing unit (CPU) consists of a control unit, ALU and registers.
These registers include
\begin{itemize}
  \item the program counter (PC) (holds the address of the next instruction),
  \item the instruction register (holds the instruction currently being decoded or executed),
  \item the address register (holds a memory address from which data will be fetched, or to which data will be returned), and
  \item the accumulator (holds temporarily the results of ALU computations).
\end{itemize}

\subsubsection{System Bus}

The system bus includes
\begin{itemize}
  \item a control bus (carries commands from the CPU and returns status signals from devices),
  \item an address bus (carries information about the device with which the CPU is communicating, namely physical memory addresses), and
  \item a data bus (carries the data being processed).
\end{itemize}

\subsubsection{Memory}

In the von~Neumann architecture, the memory contains both program instructions and data.

\subsubsection{Clock}

A clock cycle is a signal that oscillates between high and low.
It is used to coordinate actions.
The rate of the fetch-execute cycle is determined by the clock.
Instructions begin execution on the rising edge of the signal.
The time between rising edges is known as the `clock period'.
The time between the rising and falling edges of the signal is known as the `clock width'.
The number of clock cycles per second is used to determine the speed of a CPU.

Modern computers attempt to start an instruction on each clock tick.
Since each stage of the fetch-execute cycle is handled by separate circuitry, the fetch unit is freed to start the next instruction before the previous instruction is complete.
This process is known as `pipelining'.
When the pipeline is filled, five instructions are in process at a time, each at a different stage of the fetch-execute cycle
Additionally, one instruction is finished on each clock tick, making it appear as though the computer is running one instruction per tick.

\subsection{The Harvard Architecture}

The main difference between the von~Neumann and Harvard architectures is their storage of instructions and data in memory.

A von~Neumann (or `stored-program') machine stores its instructions and data in a shared memory.
This means that an instruction fetch and a data operation cannot occur at the same time because they share a common bus.
This is known as the `von~Neumann bottleneck' and can impinge on the performance of the system.
However, the architecture allows for self-modifying code that can tune its performance at runtime.

A Harvard machine stores its instructions and data in separate memories.
This makes the machine more complex, but allows instruction fetch and data operation to occur at the same time.

\subsection{MIPS R4000}

MIPS~R4000 uses the modified Harvard architecture.
Instructions and data are stored in the same memory (as in von~Neumann), but they have separate interfaces (as in Harvard).

MIPS~R4000 was one of the first \SI{64}{\bit} microprocessors.
It features integrated caches (primary on-chip and secondary off-chip), an integrated floating point unit and a deep pipeline.
It uses MIPS~III --- a compact instruction set with a regular format --- which makes it easy to convert from high-level code to machine code.

\subsubsection{Registers}

The CPU contains 32 registers denoted \texttt{\$0}--\texttt{\$31} (or \texttt{r0}--\texttt{r31}).
The registers can be either \SIlist[list-pair-separator={ or }]{32;64}{\bit} wide, depending on the mode of operation.
\texttt{\$0} (\texttt{r0}) always stores zero.
\texttt{\$31} (\texttt{r31}) is the link register used by jump and link instructions.
It should not be used by other instructions.

The ALU takes input from up to two registers and sends output to registers only.

\subsubsection{Deep Pipeline}

The typical MIPS pipeline consists of the following five stages.
\begin{enumerate}
  \item Instruction fetch (IF)
  \item Instruction decode (ID) and register fetch (RF)
  \item Execute (EX)
  \item Memory access (MEM)
  \item Write back (WB)
\end{enumerate}

MIPS~R4000 is super-pipelined; its pipeline contains eight stages rather than the regular five.
The additional stages exist due to the extension of the IF and MEM stages to account for cache overheads.
When the pipeline is full, eight instructions are in process at a time --- i.e.\ the pipeline is eight-deep.

\subsection{Assembly Language}

Binary is the only form in which a computer can be given instructions.
Computers can be programmed to translate instructions expressed in other forms into binary code.
This is known as `assembly'.

Assembly language is a human-readable alternative to machine language.
A computer can scan assembly code and convert words to binary.
The binary is assembled into instructions.

High-level languages are compiled to assembly language, which is then assembled into binary.

\subsection{Instruction Sets}

Every machine has a unique instruction set architecture (ISA).
This is the set of primitive instructions that the CPU can execute.
MIPS~R4000 uses the MIPS~III ISA.\@
Each MIPS~III instruction has a \SI{32}{\bit} representation.
It can be represented as human-readable assembly code, or binary machine code.
There are three types of MIPS~III instruction.
\begin{description}
  \item[I-type] requires one or two registers and an operand.
  \item[R-type] requires three registers.
  \item[J-type] requires an operand.
\end{description}
